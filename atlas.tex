\documentclass[a4paper,9pt]{article}
\usepackage{inputenc,graphicx}
\usepackage{caption}
\usepackage{float}
\usepackage[paperheight=12in,paperwidth=8in,top=1in,bottom=1in,right=1in,left=1in,heightrounded]{geometry}

%opening
\title{ What We learned about the ATLAS Detector}
\author{Canay Oz \\ Gamze Sokmen \\ Gokcenur Yesilyurt}

\begin{document}
 
\maketitle
 
\section{Introduction}
ATLAS brings experimental physics into new field. At the very beginning of the universe, equal amounts of matter and antimatter existed.
ATLAS is exploring the tiny difference that exists between matter and antimatter. 
\\\\
The head-on collisions of protons at its centre leave remnants that will reveal new particles and new processes in the interior of matter.
Different layers of the detector track the charged particles and measure the energies of most charged and neutral particles. 
ATLAS (and the CMS) have discovered a new particle with mass about 125 GeV, which appears to have the same properties with a Higgs boson.
So how a Higgs boson event may look in ATLAS? In this proton-proton collision event, a cluster of particles; i.e. jet, was
produced going downward, and a Higgs particle was produced going upward. 
\\\\
Moreover, ATLAS is looking for why the matter of the Universe is dominated by an unknown type of matter called dark matter.
If the components of the dark matter are new particles, ATLAS should discover them and enlighten the mystery of it. 
\\\\
Although it may have more, we can only experience the world in three dimensions.
The ATLAS experiment may see evidence that extra dimensions exist via high energy proton-proton collision events in which a gravitation
particle disappears into other dimensions. ATLAS would detect a large instability of energy in the event. 


\section{Inner Detector}
Inner detector measures the tracks of charged particles produced in LHC. Tracking detectors are formed by concentric layers. This part 
of the detector is the closest point to the particle collisions with a highest precision in the measurement. For the main security aim of the inner detector
it can be said that reducing the radiation harm which are produced by almost 350.000 particles per ${mm}^2$ per second for detector and 
readout electronic as much as possible. In addition to this, the amount of the used material is also important for the
measurement precisions. Since in order to affect the particle's trajectories in minimum level, the material quantity are kept as in
also minimum level.
\\\\
It is known that during the construction of the ATLAS detector, it was really important to consider to keep the power consumed less
and, for the data acqusition, to have more efficient speed as much as possible. For this purpose and with the reasons already
mentioned in the previous paragraph, the design of the inner detector was considered as in the order,
Pixel $\&$ Silicon Strip Detector, Transition Radiation Tracker and Central Solenoid.

\subsection{Pixel $\&$ Silicon Strip Detector}
For best resolution, high efficiency semiconductor detector elements are needed in detector technology. In this sub-part of the inner
detector, we have silicon strip detectors with 80 million tiny rectangular pixels very close to the detector layers. As it was mentioned,
silicon is a good choice as a semiconductor in order to have also a well radiation protection system. 
Pixel detector is also covered by a silicon tracker which includes a cooling sytem that keeps the temprature at -7^{o}.

\subsection{Transition Radiation Tracker}
Here, the charged particles passing through the straw can ionize the gas part of the dedector and at the same time, this produces electrical
pulses.  With the help of periods of these pulses, it is possible to find the distance between the particle's track and the wire down the 
straw's axis with a precision 0.17 $mm$. By filling with a special material in between the straw tubes cause that the pulsing electrons
to produce X-rays. Since electons are playing a big role in this process, this way is used to distinguish electons from other particles
produced during the collisions.

\subsection{Central Solenoid}
The outer layer of the inner tracker is central solenoid. Maybe for this place we can say that, here where the magic happens due to the
charge of the particles. As the central solenoid, 2 T powerful magnet, leads particles to bend their paths (tracks) according to their
charges and this helps in the same way that we said before for the CMS detector, to measure the particle's momenta.

\section{The Calorimeters}

For the ATLAS detector, the inner detector is surrounded by the calorimeters.
As the same mission in the CMS detector, calorimeters are very important
parts of the detector to absorb and measure the energies of the most
charged and neutral particles produced in LHC. The working principle 
of the calorimeters are detecting the energy deposited by the particles 
and convert them into electrical signals. These signals are able to be readable
by the data taking processes. Also it is better to indicate that, this
signals are proportional to measured energies of the particles.
The sub-parts of the calorimeters are in order, ECAL, HCAL and
endcap and forward calorimeters.

\subsection{The Electromagnetic Calorimeters}
ECAL is formed by closely spaced absorber layers of lead and liquid 
argon as the sampling material. As we understand from the name of the
calorimeter, ECAL is responsible from measuring the energies of electrons
and photons. The working principle can be approached in a similar
way with the transition radiation tracker. Here, the particles shower create
an ionization when they pass through the liquid argon. These ions are saved and read as
electric pulses. Also, having a unique accordion geometry provided to
absorber plates to cover the entire angle of the detector circularly.

\subsection{The Hadronic Calorimeters}
The particles can not be stopped by the ECAL can come untill here and
the ones are named as hadrons, leave their energies to HCAL.
Whereas for the ECAL the absorber layers are made of lead, for HCAL they are
made of steel. There are also scintillating plastic in HCAL to emit light when
charged praticles pass through them. Photomultiplier tubes convert these
light pulses to electronic signals. The scintillating tiles are oriented
radially to the beam and this helps to simplify the design of the HCAL
modules with also routing of the readout fibres. There is one more exceptional
feature of HCAL; there are only 3 optical fibres that controls 256 drawers
(the apparatus holds the PMTs and HCAL electronics).

\subsection{Endcap and Forward Calorimeters}
In order to have more absorbtion of the hadron energies, endcap and
forward calorimeters were designed for the ATLAS detector. As it was
mentioned before, there is a high radiation level region caused by proton
beams and here the argon calorimeters with the copper and tungsten absorbers
are used for HCAL measurements for radiation protection. This protective detectors extend the
acceptance of the ATLAS calorimeters to cover the almost all solid
angle around the collision point.

\section{The Muon Spectrometer}
Muon Spectrometer of Atlas is designed to measure the electrical charges and momentum of muons.
The main difference between electrons and muons is that muons are 200 times more massive.
Because of their mass and momentum they are able to pass through the calorimeter without being absorbed.
Differently from CMS, Atlas uses a second set of powerful magnets to bent muons path allowing the charges and momenta to be calculated.
\section{Data Acquisition and Computing}
There are nearly one billion proton-proton collisions per second in Atlas. It is not meaningful and
possible to store all of their to data. To select needed data Atlas uses Trigger System.
\subsection{Trigger System}
 
Atlases trigger system composed from three levels to reduce 400 million crossings per second to 20
events per second. Level-I trigger process data from every 25 ns beam crossing interval. The decision
to keep a data from an event is made less than two microseconds. This level reduce number of events
to be recorded to 100 000. Level-2 trigger analyse in greater detail and reduce number of events
to a few thousand per second. And at last Level-3 trigger performed a very detailed analysis to
reduce events to about 200 events per second. At the and of Level-3 passed events stored on data storage for offline analysis.
 
\subsection{Computing}
 
Atlas make it possible for several petabytes of data recorded, shared and analysed by physicists around
the world each year. This presents tremendous challenges in both software and hardware.
There are about 100 000 single-core processors, situated at many different locations around the world,
communicating via the new global data Grid system.




\end{document}
